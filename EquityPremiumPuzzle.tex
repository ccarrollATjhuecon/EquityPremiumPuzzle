\documentclass{handout}
\usepackage{handoutSetup}\usepackage{econtexShortcuts}  
\usepackage{econark}
\begin{document}
\handoutHeader

\begin{verbatimwrite}{\jobname.title}
  The Equity Premium Puzzle and the Riskfree Rate
\end{verbatimwrite}

\handoutNameMake


\cite{mehraPrescottPuzzle} consider a representative agent solving the joint consumption and portfolio allocation problem:
\begin{equation}\begin{gathered}\begin{aligned}
      \vFunc(m_{t}) & =  \max_{\{c_{t},\riskyshare_{t}\}} ~~\uFunc(c_{t}) + \Ex_{t}\left[\sum_{n=1}^{\infty} \Discount^{n} \uFunc(c_{t+n}) \right]
      \\  & \text{s.t.}   \nonumber
      \\      m_{t+1} & =  (m_{t}-c_{t})\Rport_{t+1} + y_{t+1}  
      \\      \Rport_{t+1} & =  \riskyshare_{t}\Risky_{t+1}+(1-\riskyshare_{t})\Rfree
    \end{aligned}\end{gathered}\end{equation}
where $\Rfree$ denotes the return on a perfectly riskless asset and $\Risky_{t+1}$ denotes the return on equities (the risky asset) held between periods $t$ and $t+1$; $\riskyshare_{t}$ is the share of end-of-period savings invested in the risky asset; $\Rport_{t+1}$ is the portfolio-weighted rate of return; and $y_{t+1}$ is noncapital income in period $t+1$.

As usual, the objective can be rewritten in recursive form:
\begin{equation}\begin{gathered}\begin{aligned}
      \vFunc(m_{t}) & =  \max_{\{c_{t},\riskyshare_{t}\}} ~~\uFunc(c_{t}) +\beta \Ex_{t}\left[\vFunc\left(\underbrace{[\riskyshare_{t}\Risky_{t+1}+(1-\riskyshare_{t})\Rfree]}_{{\Rport}_{t+1}}(m_{t}-c_{t})+{y}_{t+1}\right)\right]
    \end{aligned}\end{gathered}\end{equation}

The first order condition with respect to $c_{t}$ is
\begin{equation}\begin{gathered}\begin{aligned}
      \uFunc^{\prime}(c_{t}) & =  \beta \Ex_{t}[ \Rport_{t+1}\vFunc^{\prime}({m}_{t+1})] \label{eq:cfoc}.
    \end{aligned}\end{gathered}\end{equation}
and, taking $\mNrm$ and $\cNrm$ as given, the FOC with respect to $\riskyshare_{t}$ is
\begin{equation}\begin{gathered}\begin{aligned}
      \Ex_{t}[(\Risky_{t+1}-\Rfree)\vFunc^{\prime}({m}_{t+1})({m}_{t}-{c}_{t})] & =  0 
      \\      \Ex_{t}[(\Risky_{t+1}-\Rfree)\vFunc^{\prime}({m}_{t+1})\phantom{({m}_{t}-{c}_{t})}] & =  0 \label{eq:gamfoc}
      .
    \end{aligned}\end{gathered}\end{equation}

But the usual logic of the \handoutC{Envelope} theorem tells us that 
\begin{equation}\begin{gathered}\begin{aligned}
      \uFunc^{\prime}(c_{t+1}) & =  \vFunc^{\prime}(m_{t+1}), \label{eq:envelope}
    \end{aligned}\end{gathered}\end{equation}
so, substituting (\ref{eq:envelope}) into (\ref{eq:cfoc}) and (\ref{eq:gamfoc}), the above FOC's reduce to
\begin{equation}\begin{gathered}\begin{aligned}
      \uFunc^{\prime}(c_{t}) & =  \Ex_{t}\left[\beta  \Rport_{t+1} \uFunc^{\prime}({c}_{t+1})\right] \label{eq:ceuler}
      \\      \Ex_{t}[(\Risky_{t+1}-\Rfree)\uFunc^{\prime}({c}_{t+1})] & =  0 %\label{eq:gameuler}
    \end{aligned}\end{gathered}\end{equation}

Now assume CRRA utility, $\uFunc(c) = c^{1-\CRRA}/(1-\CRRA)$ and divide both 
sides by $c_{t}^{-\CRRA}$ to get
\begin{equation}\begin{gathered}\begin{aligned}
      \Ex_{t}[(c_{t+1}/c_{t})^{-\CRRA}(\Risky_{t+1}-\Rfree)] & =  0. \label{eq:gameulernew}
    \end{aligned}\end{gathered}\end{equation}


\noindent Now recall the following two facts:

\begin{quote}
  \begin{itemize}

  \item[Fact 1: ] If $\Delta c_{t+1}/c_{t}$ is small, $c_{t+1}/c_{t} \approx 1+ \Delta \log c_{t+1}.$

  \item[Fact 2: ] If $z$ is small, $(1+z)^{\lambda} \approx 1 + \lambda z$.

  \end{itemize}
\end{quote}

\noindent Using these two facts, equation (\ref{eq:gameulernew}) can be approximated by
\begin{equation}\begin{gathered}\begin{aligned}
      \Ex_{t}[(1-\CRRA \Delta \log {c}_{t+1})(\Risky_{t+1}-\Rfree)] & \approx  0.
    \end{aligned}\end{gathered}\end{equation}

\noindent Using one more fact,
\begin{quote}
  \begin{itemize}
  \item[Fact 3:] $\Ex [xy] = \Ex [x]\Ex [y] + \text{cov}(x,y),$
  \end{itemize}
\end{quote}
we get
\begin{equation}
  (1-\CRRA \Ex_{t}[\Delta \log {c}_{t+1}])(\Rfree - \Ex_{t}[\Risky_{t+1}])+\text{cov}_{t}(-\CRRA \Delta \log {c}_{t+1},-\Risky_{t+1})  \approx  0  
\end{equation}
or
\begin{equation}\begin{gathered}\begin{aligned}
      \Ex_{t}[\Risky_{t+1}]-\Rfree & \approx  \frac{\CRRA \text{cov}_{t}(\Delta \log {c}_{t+1},\Risky_{t+1})}{1-\CRRA \Ex_{t}[ \Delta \log {c}_{t+1}]} 
      \\                             & \approx  \CRRA \text{cov}_{t}(\Delta \log {c}_{t+1},\Risky_{t+1}) \label{eq:eqprem}
    \end{aligned}\end{gathered}\end{equation}
\noindent where the last approximation holds because $\Ex_{t}[\Delta 
\log {c}_{t+1}]$ is small.  

\vspace{.25in}

\noindent {\it \bf The Equity Premium Puzzle}


Because this expression must hold at all $t$, we can check it 
empirically by calculating empirical estimates of the two components 
and assuming that the sample averages correspond to the representative 
agent's expectations.  That is, if we have data for periods $1 \ldots 
n $, we assume that the unconditional expectations correspond to the 
sample means, $\Ex [\Risky] = (1/n) \sum_{s=1}^{n} \Risky_{s}$; $\Ex [\Delta 
\log c] = (1/n) \sum_{s=1}^{n} \Delta \log c_{s}$; and 
$\text{cov}(\Delta \log c,\Risky) = (1/n) 
\sum_{s=1}^{n} (\Delta \log c_{s} - \Ex [\Delta \log 
c])(\Risky_{s}-\Ex [\Risky]).$

The equity premium puzzle is essentially that $\text{cov}(\Delta \log 
c,\Risky)$ is very small (about 0.004) but $\Ex [\Risky]-\Rfree$ is about 0.08 
(stocks have earned real returns of about 8 percent more than riskless 
assets over the historical period), which means that the only way 
equation (\ref{eq:eqprem}) can hold is if $\CRRA$ is implausibly large 
(these values imply a value of $\CRRA=20$).

How do we know what plausible values of $\CRRA$ are?  Consider the 
following.  You must choose between a gamble in which you consume 
\$50,000 for the rest of your life with probability 0.5 and \$100,000 
with probability 0.5, or consuming some amount $X$ with certainty.  
The coefficient of relative risk aversion determines the $X$ which 
would make you indifferent between consuming X or being exposed to the 
gamble.  For example, if $\CRRA = 0$, then you have no risk aversion at 
all and you will be indifferent between \$75,000 with certainty and 
the 50/50 gamble with expected value of \$75,000.  Here are the values 
of X associated with different values of $\CRRA$ (table taken from 
Mankiw and Zeldes~\cite{mankiw&zeldes:stockholders}.)

\vspace{.25in}

\begin{table}
  \center
  \begin{tabular}{|c|c|}
    \hline  $\CRRA$  & $     X $ 
    \\ \hline
    1       &       70,711
    \\      3       &       63,246
    \\      5       &       58,565
    \\      10      &       53,991
    \\      20      &       51,858
    \\      30      &       51,209
    \\      $\infty$        &       50,000 
    \\ \hline
  \end{tabular}
\end{table}

% \vspace{.25in}
\noindent {\it \bf The Riskfree Rate Puzzle}

Rewrite the consumption Euler equation (\ref{eq:ceuler}) as 
\begin{equation}\begin{gathered}\begin{aligned}
      \uFunc^{\prime}(c_{t}) & =  \Ex_{t}\left[\beta (\Rfree  + \riskyshare_{t} [\Risky_{t+1} - \Rfree])\uFunc^{\prime}({c}_{t+1})\right] \label{eq:newgam}
    \end{aligned}\end{gathered}\end{equation}
and note that from (\ref{eq:gameulernew}) we know that $\Ex_{t}[\beta 
\riskyshare_{t}(\Risky_{t+1}-\Rfree)\uFunc^{\prime}({c}_{t+1})] = 0$ so that 
(\ref{eq:newgam}) reduces to the ordinary Euler equation
\begin{equation}\begin{gathered}\begin{aligned}
      \uFunc^{\prime}(c_{t}) & =  \Ex_{t}[\beta \Rfree \uFunc^{\prime}({c}_{t+1})]
      \\  1 & =  \beta \Rfree \Ex_{t}[ ({c}_{t+1}/c_{t})^{-\CRRA}].
    \end{aligned}\end{gathered}\end{equation}
Using the same `facts' and approximations as above, we get the 
standard approximation to the Euler equation,
\begin{equation}\begin{gathered}\begin{aligned}
      \Delta \log c_{t+1} & \approx  (1/\CRRA) (\rfree - \timeRate).
    \end{aligned}\end{gathered}\end{equation}

The `riskfree rate puzzle' is that average consumption growth per \opt{MarginNotes}{\marginpar{\tiny In principle, riskfree rate puzzle might be explicable by overlapping generations.  In practice, it's hard to make this work well.  Note also that a precautionary saving motive reduces the riskfree rate puzzle by adding the variance term to consumption growth.}}
capita has been about 1.5 percent (in the US in the postwar period) 
while real riskfree interest rates have been at most 1 percent.  Even
if we assume a time preference rate of $\timeRate=0$ (no impatience 
at all, e.g. $\beta=1$), the only way this equation can hold is if $\CRRA$ is a
very small number (maybe even less than one).  Of course, this
is precisely the opposite of the conclusion of the equity premium
puzzle, which implies the $\CRRA$ must be very large.


\pagebreak
\centerline{\Large Appenix}
\appendix
\section{Solution for $t$ when $\cFunc_{t+1}$ is Known}

In either the life cycle version of the model or an infinite horizon model that is being solved by time iteration, the consumption function in $t+1$ will be known.

Designating that consumption function as $\cFunc_{t+1}(\mNrm)$, with derivative $\cFunc^{\prime}_{t+1}(\mNrm)$, we can derive an approximation to the optimal portfolio share as follows.

\newcommand{\ERport}{\bar{\Rport}}
\newcommand{\cNxt}{\bar{\cFunc}_{t+1}}
\newcommand{\orderTwo}{+(1/2)(\riskyshare\EpremLog_{t+1}{a}_{t})^{2}\cFunc^{\prime\prime}_{t+1}}
\renewcommand{\orderTwo}{}

First define $\ERport_{t+1}(\riskyshare)=\Ex_{t}[\Rport_{t+1}]$ and define consumption at the expectation of the portfolio return as
\begin{align}
\bar{\cNrm}_{t+1} &= \cNrm_{t+1}(\ERport_{t+1}{a}_{t} + \yNrm_{t+1})
\\ \bar{\cNrm}^{\prime}_{t+1}& = \cNrm_{t+1}^{\prime}(\ERport_{t+1}{a}_{t} + \yNrm_{t+1})
\end{align}

For simplicity, we will henceforth assume that $\yNrm_{t+1}=1$.  Results below all go through for the case where $\Ex_{t}[\yNrm_{t+1}]=1$.

Using the fact that for a portfolio share of $\riskyshare$ the realized return premium will $\Rport_{t+1}=(\Risky_{t+1}-\Rfree)\riskyshare$, or (calling the realized equity premium $\EpremLog_{t+1}=(\Risky_{t+1}-\Rfree),$ start with the FOC for $\riskyshare$:
\begin{equation}\begin{gathered}\begin{aligned}
      \Ex_{t}[(\cFunc_{t+1}(\Rport_{t+1}{a}_{t}+1))^{-\CRRA}\EpremLog_{t+1}] & =  0 \label{eq:gameulernewnew}
      \\ \Ex_{t}[(\cFunc_{t+1}((\Rport_{t+1}-\ERport_{t+1}+\ERport_{t+1}){a}_{t}+1))^{-\CRRA}\EpremLog_{t+1}] & =  0
      \\ \Ex_{t}[(\cFunc_{t+1}((\ERport_{t+1}+\riskyshare\EpremLog_{t+1}){a}_{t}+1))^{-\CRRA}\EpremLog_{t+1}] & =  0
      \\ \Ex_{t}[(\cNxt+\riskyshare\EpremLog_{t+1}\bar{\cFunc}^{\prime}_{t+1}{a}_{t}\orderTwo )^{-\CRRA}\EpremLog_{t+1}] & \approx  0
      % \\ \Ex_{t}[(\cNxt+\riskyshare\EpremLog_{t+1}\bar{\cFunc}^{\prime}_{t+1}{a}_{t}\orderTwo )^{-\CRRA}\EpremLog_{t+1}] & \approx  0
      \\ (\cNxt)^{-\CRRA}
      \Ex_{t}[
      (1+
      \cNxt^{-1}(\riskyshare\EpremLog_{t+1}\bar{\cFunc}^{\prime}_{t+1}{a}_{t}\orderTwo)
      )^{-\CRRA}\EpremLog_{t+1}
      ]
      & \approx  0
      \\ \Ex_{t}[(1+\cNxt^{-1}(\riskyshare\EpremLog_{t+1}\bar{\cFunc}^{\prime}_{t+1}{a}_{t}\orderTwo))^{-\CRRA}\EpremLog_{t+1}]& \approx  0
      \\ 
      \Ex_{t}[
      (1{-\CRRA}
      \cNxt^{-1}(\riskyshare\EpremLog_{t+1}\bar{\cFunc}^{\prime}_{t+1}{a}_{t}\orderTwo)
      )\EpremLog_{t+1}
      ]
      & \approx  0
    \end{aligned}\end{gathered}\end{equation}
where the last step uses $(1+\epsilon)^{\delta} \approx (1+\delta \epsilon)$.

Now define the proportional MPC out of an additional unit of portfolio return as
\begin{equation}\begin{gathered}\begin{aligned}
      \kappa & = \bar{\cFunc}^{\prime}_{t+1}{a}_{t}\orderTwo\cNxt^{-1}
    \end{aligned}\end{gathered}\end{equation}
and recalling that $\EpremLog \equiv \Ex_{t}[\EpremLog_{t+1}]=-\sigma^{2}_{\EpremLog}/2$ substitute this into the foregoing to obtain
\begin{equation}\begin{gathered}\begin{aligned}
      \Ex_{t}[
      (1{-\CRRA}
      \left(
        \kappa\riskyshare \EpremLog_{t+1}
      \right)
      )\EpremLog_{t+1}
      ]
      & \approx  0
      \\ 
      \Ex_{t}[
      (1{-\CRRA}
      \left(
        \kappa\riskyshare \EpremLog_{t+1}
      \right)
      )]\Ex_{t}[\EpremLog_{t+1}]+\cov({-\CRRA}\kappa\riskyshare \EpremLog_{t+1},\EpremLog_{t+1})
      & \approx  0
      \\ 
      (1-\CRRA \kappa\riskyshare \EpremLog)\EpremLog+\cov({-\CRRA}\kappa\riskyshare \EpremLog_{t+1},\EpremLog_{t+1})
      & \approx  0
      \\ 
      (1-\CRRA \kappa\riskyshare \EpremLog) \EpremLog
      & \approx  {\CRRA}\kappa\riskyshare \sigma^{2}_{\EpremLog}
      \\ 
      \EpremLog
      & \approx  \riskyshare \left({\CRRA} \sigma^{2}_{\EpremLog}+\CRRA \EpremLog^{2}\right)\kappa
      \\
      \left(\frac{ \EpremLog}{{\CRRA}( \sigma^{2}_{\EpremLog}+ \EpremLog^{2})\kappa}\right)
      & \approx  \riskyshare 
    \end{aligned}\end{gathered}\end{equation}


\clearpage 

\input handoutBibMake



\end{document}
